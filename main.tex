\documentclass{article}
\pdfoutput=1
\usepackage{import}
\usepackage{graphicx}
\usepackage{amsthm}
\usepackage{amsfonts}
\usepackage{enumitem}
\usepackage{amsmath}
\usepackage{amssymb}
\usepackage{hyperref}
\usepackage{url}
\usepackage{pst-node}
\usepackage{tikz-cd}  	
\usepackage[all]{xy}
\usepackage{tikz}
\usepackage{xcolor}
\usepackage{stmaryrd}
\usepackage{enumitem}
\usepackage{rotating}
\newcommand*{\isoarrow}[1]{\arrow[#1,"\rotatebox{90}{\(\sim\)}"
]}
\usetikzlibrary{matrix}
\usepackage{etoolbox}

\hypersetup{
    colorlinks=true,
    linkcolor=blue,
    filecolor=magenta,      
    urlcolor=cyan,
    pdftitle={Overleaf Example},
    pdfpagemode=FullScreen,
    }
    
    
\theoremstyle{definition}
\newtheorem{theorem}{Theorem}[section]
\newtheorem{fact}[theorem]{Fact}
\newtheorem{proposition}[theorem]{Proposition}
\newtheorem{lemma}[theorem]{Lemma}
\newtheorem{corollary}[theorem]{Corollary}
\newtheorem{exercise}[theorem]{Exercise}
\newtheorem{formula}[theorem]{Formula}
\newtheorem{definition}[theorem]{Definition}
\newtheorem{example}[theorem]{Example}
\newtheorem{examples}[theorem]{Examples}
\newtheorem{remark}[theorem]{Remark}
\newtheorem{convention}[theorem]{Convention}
\DeclareMathOperator*{\supp}{supp}
\DeclareMathOperator{\Hom}{{\text{Hom}}}
\DeclareMathOperator{\IC}{{\textbf{IC}}}
\DeclareMathOperator{\Ext}{{\text{Ext}}}
\DeclareMathOperator{\End}{{\text{End}}}
\DeclareMathOperator{\RHom}{{\text{RHom}}}
\DeclareMathOperator{\Diff}{{\text{Diff}}}
\DeclareMathOperator{\RSHom}{{\mathcal{RH}\text{om}}}
\DeclareMathOperator{\ESxt}{{\mathcal{E}\text{xt}}}
\DeclareMathOperator{\REnd}{{\text{REnd}}}
\DeclareMathOperator{\Ker}{{\text{Ker}}}
\DeclareMathOperator{\Image}{{\text{Im}}}
\DeclareMathOperator{\Spec}{{\text{Spec}}}
\DeclareMathOperator{\ddeg}{{\text{deg}}}
\DeclareMathOperator{\Hilb}{{\text{Hilb}}}
\DeclareMathOperator{\Sym}{{\text{Sym}}}




\DeclareMathOperator{\heart}{\ensuremath\heartsuit}
\DeclareMathOperator{\Id}{Id} 
\newcommand{\Modu}{{\text{-mod}}}

\title{$HH_0(D_q(T)^W)$}


\begin{document}

\maketitle

There are two ways of looking at this, one way is by the localisation theorem of Nevins and Mcgerty \cite{article1}, which says:
\begin{theorem}
If $A$ is a quantisation of $\mathcal{O}(X)$, $\rho:\tilde{X}\to X$ is a symplectic resolution and $\mathcal{A}$ is a quantisation of $\mathcal{O}(\tilde{X})$, moreover $A$ has finite dimension, then $A\Modu\cong \mathcal{A}\Modu$. 
\end{theorem}

As Hochschild homology is a Morita invariant, we can look at $HH_i(\mathcal{A})$, which by Nest--Tsygan is $H^{\dim X -i}(\tilde{X})$ as $\tilde{X}$ is smooth symplectic. In particular, $HH_0$ is given by the top cohomology.\\

Or we can look at the Hochschild--de Rham picture: 
\begin{theorem}
$HH^{dR}_i(A)\cong H^{\dim X -i}(\tilde{X})$, assuming that the conjecture $\rho_*\Omega_{\tilde{X}}\cong M_X$ is true,
\end{theorem}
but this doesn't assume $A$ is finite dimensional. And $HH^{dR}_0\cong HH_0$. I think the conjecture is true in the type A B C cases as it is true in the $\Hilb$ case.\\

Nevertheless, we need to compute the top cohomology of the resolution.\\

Let $T\cong \mathbb{C}^*\times\mathbb{C}^*$ be the 2-torus. Assume that we are in type A, $W\cong S_n$. We are interested in founding the resolution $\widetilde{T^{n-1}/S_n}$ of $T^{n-1}/S_n$, where $T^{n-1}$ sits in $T^n$ as $((a_1,b_1),\dots,(a_{n-1}, b_{n-1}),(a_1^{-1}\dots a_{n-1}^{-1}, b_1^{-1}\dots b_{n-1}^{-1}))$ (the reflection representation). \\

Let $C_n$ be the cyclic group of order $n$. There is a $C_n\times C_n$-covering map $f:T^{n-1}/S_n\times T\to T^n/S_n$:

$$\{(a_1,b_1),\dots,(a_{n-1}, b_{n-1})\};(c,d)\mapsto \{(a_1c,b_1d),\dots,(a_{n-1}c, b_{n-1}d),(a_1^{-1}\dots a_{n-1}^{-1}c, b_1^{-1}\dots b_{n-1}^{-1}d)\}$$

The $C_n\times C_n$ action, amounts to the choice of the $n$-th roots of $a_1\dots a_n$ and $b_1 \dots b_n$, and the action is given by 
$$(x,y)\cdot (\{(a_1,b_1),\dots,(a_{n-1}, b_{n-1})\};(c,d))=\{(xa_1,yb_1),\dots,(xa_{n-1}, yb_{n-1})\};(x^{-1}c,y^{-1}d).$$

Moreover, $f$ respects this action, with trivial action on the base.\\

As it is a covering, the resolution should lift (Symplectic resolution should respect (\'etale) base change (covering is \'etale)):

\[\begin{tikzcd}
	{\text{Resolution}} & {\Hilb^n (T)} \\
	{T^{n-1}/S_n\times T} & {T^{n}/S_n},
	\arrow["{\tilde{f}}", from=1-1, to=1-2]
	\arrow["\rho", from=1-2, to=2-2]
	\arrow[from=1-1, to=2-1]
	\arrow["f"', from=2-1, to=2-2]
\end{tikzcd}\]

Moreover the map $\tilde{f}$ should also be a covering.\\

The map $\rho$ is a symplectic resolution as we are in type A (more generally works in type B C).\\

So the resolution looks like $\widehat{\Hilb^n(T)}:=\Hilb^n(T)\times_{T^n/S_n} T^{n-1}/S_n\times T$, this thing is symplectic as it is a covering of a smooth symplectic thing.\\

The cohomology of $\widehat{\Hilb^n(T)}$ is related to the cohomology of the resolution $\widetilde{T^{n-1}/S_n}$ of $T^{n-1}/S_n$ by the K\"unneth formula. Let $h_i$ be the betti number of $\widetilde{T^{n-1}/S_n}$. Then $$\dim H^{2n}(\widehat{\Hilb^n(T)})=h_{2n-2},$$
$$\dim H^{2n-1}(\widehat{\Hilb^n(T)})=2h_{2n-2}+h_{2n-3}, $$
$$\dim H^{i}(\widehat{\Hilb^n(T)})=h_i+2h_{i-1}+h_{i-2} \text{ for } 2\leq i\leq 2n-2,$$
$$\dim H^{1}(\widehat{\Hilb^n(T)})=h_1+2$$
\\

Therefore we just need to calculate $\dim H^{i}(\widehat{\Hilb^n(T)}).$\\

According to \cite[Corollary 3]{article2} (proved by generalising Nakajima's result), we can calculate the cohomology using the following. 

$$\bigoplus_{n\geq 0}H^*(\widehat{\Hilb^n(T)},\mathbb{C}[2n])\cong \bigoplus_{\chi\in (C_n\times C_n)^\vee}\Sym(\bigoplus_{\nu\geq 1}H^*(T,L^\nu_\chi[2]))$$

There is an isomorphism of bi-graded vector spaces, where the first grading is by the cohomology degree, and the second grading (called weighting) is the number of points $n$ on the left and $H^*(T,L^\nu_\chi)$ has weight $\nu$.\\

From this, we can compute all cohomologies of $\widehat{\Hilb^n(T)}$. Note that 
\begin{align*}
    H^*(T,L^\nu_\chi[2])&\cong H^*(S^1\times S^1,L^\nu_{\chi_1}[1]\boxtimes L^\nu_{\chi_2}[1])\\
    &\cong H^*(S^1,L^\nu_{\chi_1}[1]) \otimes H^*(S^1,L^\nu_{\chi_2}[1])\\
    &\cong \delta_{\chi_1^\nu,1}\delta_{\chi_2^\nu,1}(\mathbb{C}[0]\oplus \mathbb{C}[-1])^{\otimes 2} \{\nu\}
\end{align*}

Here we are using $[-]$ for (cohomological) grading and $\{-\}$ for weighting.\\

Identifying $(C_n \times C_n)^\vee\cong \mathbb{Z}/n\mathbb{Z}\times \mathbb{Z}/n\mathbb{Z}$ and rewriting $\chi=(\chi_1,\chi_2)$ as $(a,b)$. The RHS becomes $$\bigoplus_{(a,b)\in (\mathbb{Z}/n\mathbb{Z}\times \mathbb{Z}/n\mathbb{Z})}\Sym(\bigoplus_{\nu\geq 1}\delta_{a\nu,0}\delta_{b\nu,0}(\mathbb{C}[0]\oplus \mathbb{C}[-1])^{\otimes 2}\{\nu\})$$


\begin{align*}
    \delta_{a\nu,0}\delta_{b\nu,0}\text{ is non-zero }&\text{if and only if }n|a\nu\text{ and }n|b\nu\\
    &\text{if and only if }\frac{n}{\gcd(a,n)}|\nu\text{ and }\frac{n}{\gcd(b,n)}|\nu\\
    &\text{if and only if }\frac{n}{\gcd(a,b,n)}|\nu
\end{align*}

(Taking $\gcd(0,n)=n$, and $\gcd(0,a,n)=\gcd(a,n)$.)\\

So the RHS becomes
$$\bigoplus_{(a,b)\in (\mathbb{Z}/n\mathbb{Z}\times \mathbb{Z}/n\mathbb{Z})}\Sym(\bigoplus_{k\geq 1}(\mathbb{C}[0]\oplus \mathbb{C}^2[-1]\oplus\mathbb{C}[-2])\{k\frac{n}{\gcd(a,b,n)}\}).$$

%$\Sym(\bigoplus_{k\geq 1}(\mathbb{C}[0]\oplus \mathbb{C}^2[-1]\oplus\mathbb{C}[-2])$ has Hilbert series $$\frac{t}{(1-t)^2(1-t^2)}=t + 2 t^2 + 4 t^3 + 6 t^4 + 9 t^5 + 12 t^6 + 16 t^7 + 20 t^8 + 25 t^9 + \dots$$







For the top cohomology, we have:

$$\bigoplus_{n\geq 0}H^{2n}(\widehat{\Hilb^n(T)},\mathbb{C})\cong \bigoplus_{(a,b)\in (\mathbb{Z}/n\mathbb{Z}\times \mathbb{Z}/n\mathbb{Z})}\Sym(\bigoplus_{k\geq 1}\mathbb{C}\{k\frac{n}{\gcd(a,b,n)}\}).$$

Let $\mathcal{P}(n)$ be the number of partitions of $n$. Taking weight $n$, we get that 

\begin{align*}
    \dim H^{2n}(\widehat{\Hilb^n(T)})&=\sum_{(a,b)\in (\mathbb{Z}/n\mathbb{Z}\times \mathbb{Z}/n\mathbb{Z})} \mathcal{P}(\gcd(a,b,n))\\
    &=\sum_{d|n} \mathcal{P}(d)J_2(\frac{n}{d}),
\end{align*}

where $J_2$ is the second \href{https://en.wikipedia.org/wiki/Jordan\%27s_totient_function}{Jordan's totient function}.

For $n=p$ a prime, this number is $\mathcal{P}(p)+p^2-1$. When $p=2$, this gives $2+4-1=5$.\\

Let $P_r(m)$ be the \emph{set} of $r$-component multipartitions of $m$, that is, an $r$-tuple of partitions $\lambda(1),\dots,\lambda(r)$ where each $\lambda(i)$ is a partition of some $a_i$ and the $a_i$ sum to $m$. Let $\mathcal{P}_r(m)$ be the size of this set.\\

We are interested in 4-component multipartitions, inside of ${P}_4(m)$, define the subset ${P}_4^{(0,1,1,2)}(m,i)$ consisting elements $(\lambda(1),\lambda(2),\lambda(3),\lambda(4))$ such that $|\lambda(2)|+|\lambda(3)|+2|\lambda(4)|=i$. Let $\mathcal{P}_4^{(0,1,1,2)}(m,i)$ be the \emph{size} of this subset.

In general, for the $(2n-i)$th cohomology, $\dim H^{2n-i}(\widehat{\Hilb^n(T)})$ has the following description:

$$\dim H^{2n-i}(\widehat{\Hilb^n(T)})=\sum_{(a,b)\in (\mathbb{Z}/n\mathbb{Z}\times \mathbb{Z}/n\mathbb{Z})} \mathcal{P}_4^{(0,1,1,2)}(\gcd(a,b,n),i).$$

%\begin{equation*}
%\dim H^{2n-i}(\widehat{\Hilb^n(T)})=\sum_{(a,b)\in (\mathbb{Z}/n\mathbb{Z}\times \mathbb{Z}/n\mathbb{Z})} \#\begin{Bmatrix}
%\alpha_1,\dots \alpha_{k_1}, \beta_1,\dots \beta_{k_2}, \gamma_1,\dots \gamma_{k_3}\in \mathbb{Z}_+\\
%x_1,\dots x_{k_1}, y_1,\dots y_{k_2}, z_1,\dots z_{k_3}\in \mathbb{Z}_+ \\
%\begin{split}
%\text{such that }&
%\begin{split}
%&\alpha_1+\dots+\alpha_{k_1}+\\
%&\beta_1+\dots+\beta_{k_2}+\\
%2&\gamma_1+\dots+2\gamma_{k_3}=i
%\end{split}\\
%\text{and } &\begin{split}
%&\alpha_1 x_1+\dots+\alpha_{k_1}x_{k_1}+\\
%&\beta_1 y_{1}+\dots+\beta_{k_2}y_{k_2}+\\
%&\gamma_1z_1+\dots+\gamma_{k_3}z_{k_3}=\gcd(a,b,n)
%\end{split}
%\end{split}
%\end{Bmatrix}.
%\end{equation*}



\bibliographystyle{alpha}
{\footnotesize
\bibliography{bib}}

\end{document}
